%%%%%%%%%%%%%%%%%%%%%%%%%%%%%%%%%%%%%%%%%
% Masters/Doctoral Thesis 
% LaTeX Template
% Version 2.5 (27/8/17)
%
% This template was downloaded from:
% http://www.LaTeXTemplates.com
%
% Version 2.x major modifications by:
% Vel (vel@latextemplates.com)
%
% This template is based on a template by:
% Steve Gunn (http://users.ecs.soton.ac.uk/srg/softwaretools/document/templates/)
% Sunil Patel (http://www.sunilpatel.co.uk/thesis-template/)
%
% Template license:
% CC BY-NC-SA 3.0 (http://creativecommons.org/licenses/by-nc-sa/3.0/)
%
%%%%%%%%%%%%%%%%%%%%%%%%%%%%%%%%%%%%%%%%%

%----------------------------------------------------------------------------------------
%	PACKAGES AND OTHER DOCUMENT CONFIGURATIONS
%----------------------------------------------------------------------------------------

\documentclass[
11pt, % The default document font size, options: 10pt, 11pt, 12pt
%oneside, % Two side (alternating margins) for binding by default, uncomment to switch to one side
english, % ngerman for German
singlespacing, % Single line spacing, alternatives: onehalfspacing or doublespacing
%draft, % Uncomment to enable draft mode (no pictures, no links, overfull hboxes indicated)
%nolistspacing, % If the document is onehalfspacing or doublespacing, uncomment this to set spacing in lists to single
%liststotoc, % Uncomment to add the list of figures/tables/etc to the table of contents
%toctotoc, % Uncomment to add the main table of contents to the table of contents
%parskip, % Uncomment to add space between paragraphs
%nohyperref, % Uncomment to not load the hyperref package
headsepline, % Uncomment to get a line under the header
%chapterinoneline, % Uncomment to place the chapter title next to the number on one line
%consistentlayout, % Uncomment to change the layout of the declaration, abstract and acknowledgements pages to match the default layout
]{MastersDoctoralThesis} % The class file specifying the document structure

\usepackage[utf8]{inputenc} % Required for inputting international characters
\usepackage[T1]{fontenc} % Output font encoding for international characters

\usepackage{mathpazo} % Use the Palatino font by default

\usepackage[backend=biber,style=authoryear,natbib=true]{biblatex} % Use the bibtex backend with the authoryear citation style (which resembles APA)

\addbibresource{example.bib} % The filename of the bibliography

\usepackage[autostyle=true]{csquotes} % Required to generate language-dependent quotes in the bibliography

%----------------------------------------------------------------------------------------
%	MARGIN SETTINGS
%----------------------------------------------------------------------------------------

\geometry{
	paper=a4paper, % Change to letterpaper for US letter
	inner=3.8cm, % Inner margin
	outer=1.9cm, % Outer margin
	bindingoffset=.5cm, % Binding offset
	top=2cm, % Top margin
	bottom=2cm, % Bottom margin
	%showframe, % Uncomment to show how the type block is set on the page
}

%----------------------------------------------------------------------------------------
%	THESIS INFORMATION
%----------------------------------------------------------------------------------------

\thesistitle{UAV Smart Wing - a Control Systems study} % Your thesis title, this is used in the title and abstract, print it elsewhere with \ttitle
\supervisor{Dr. T. Glyn \textsc{Thomas}} % Your supervisor's name, this is used in the title page, print it elsewhere with \supname
\examiner{} % Your examiner's name, this is not currently used anywhere in the template, print it elsewhere with \examname
\degree{MEng Aeronautics and Astronautics} % Your degree name, this is used in the title page and abstract, print it elsewhere with \degreename
\author{Bhagyesh \textsc{Govilkar}} % Your name, this is used in the title page and abstract, print it elsewhere with \authorname
\addresses{} % Your address, this is not currently used anywhere in the template, print it elsewhere with \addressname


\keywords{} % Keywords for your thesis, this is not currently used anywhere in the template, print it elsewhere with \keywordnames
\university{\href{http://www.southampton.ac.uk}{University of Southampton}} % Your university's name and URL, this is used in the title page and abstract, print it elsewhere with \univname
\department{\href{http://department.university.com}{Department or School Name}} % Your department's name and URL, this is used in the title page and abstract, print it elsewhere with \deptname
\group{\href{http://researchgroup.university.com}{Research Group Name}} % Your research group's name and URL, this is used in the title page, print it elsewhere with \groupname
\faculty{\href{http://faculty.university.com}{Faculty Name}} % Your faculty's name and URL, this is used in the title page and abstract, print it elsewhere with \facname

\AtBeginDocument{
\hypersetup{pdftitle=\ttitle} % Set the PDF's title to your title
\hypersetup{pdfauthor=\authorname} % Set the PDF's author to your name
\hypersetup{pdfkeywords=\keywordnames} % Set the PDF's keywords to your keywords
\hypersetup{allcolors=.}
}

\begin{document}

\frontmatter % Use roman page numbering style (i, ii, iii, iv...) for the pre-content pages

\pagestyle{plain} % Default to the plain heading style until the thesis style is called for the body content

%----------------------------------------------------------------------------------------
%	TITLE PAGE
%----------------------------------------------------------------------------------------

\begin{titlepage}
\begin{center}

\vspace*{.06\textheight}
{\scshape\LARGE \univname\par}\vspace{1.5cm} % University name

\HRule \\[0.4cm] % Horizontal line
{\huge \bfseries \ttitle\par}\vspace{0.4cm} % Thesis title
\HRule \\[1.5cm] % Horizontal line
 
\begin{minipage}[t]{0.4\textwidth}
\begin{flushleft} \large
\emph{Author:}\\
{\authorname} % Author name - remove the \href bracket to remove the link
\end{flushleft}
\end{minipage}
\begin{minipage}[t]{0.4\textwidth}
\begin{flushright} \large
\emph{Supervisor:} \\
{\supname} % Supervisor name - remove the \href bracket to remove the link  
\end{flushright}
\end{minipage}\\[3cm]
\begin{center}
\includegraphics[scale=0.3]{crest.jpg}		% University/department logo - uncomment to place it
\end{center}

\large \textit{This report is submitted in partial fulfillment of the
requirements for the MEng Aeronautics and Astronautics, Faculty of Engineering and the
Environment, University of Southampton}\\[0.3cm] % University requirement text

 
\vfill

{\large \today}\\[4cm] % Date

 
\vfill
\end{center}
\end{titlepage}

%----------------------------------------------------------------------------------------
%	DECLARATION PAGE
%----------------------------------------------------------------------------------------

\section*{Declaration}
 I, Bhagyesh Govilkar declare that this thesis and the work presented in it are my own and has
been generated by me as the result of my own original research.
I confirm that:
\begin{enumerate}
\item  This work was done wholly or mainly while in candidature for a degree at this
University;
\item  Where any part of this thesis has previously been submitted for any other
qualification at this University or any other institution, this has been clearly stated;
\item  Where I have consulted the published work of others, this is always clearly
attributed;
\item  Where I have quoted from the work of others, the source is always given. With
the exception of such quotations, this thesis is entirely my own work;
\item  I have acknowledged all main sources of help;
\item  Where the thesis is based on work done by myself jointly with others, I have
made clear exactly what was done by others and what I have contributed myself;
\item  None of this work has been published before submission.
\end{enumerate}
 

\cleardoublepage


\section*{Abstract}
Unmanned Aerial Vehicles are becoming increasingly mainstream. With faster and more efficient ways of manufacturing such as 3D printing gaining momentum, this will only serve to bolster the emerging market. The University of Southampton's SULSA project (Southampton University Laser Sintered Aircraft) produced an UAV that was made entirely from 3D printed parts. 

There are several contemporary and potential applications of UAVs: they are widely used in agriculture to give farmers a detailed picture of the health of their farm, the amount of resources such as pesticides required and they can help cut down on labour costs; they are already being used in the defence sector mostly to carry out surveillance and reconnaissance activities, a handful are being used for offensive operations such as the MQ-9 Reaper. 

With this increasing usage of UAVs, there is a strong need for a control system that can give the user a stable, reliable and robust aircraft. The aim is to reduce the frequency of UAVs crashing, increase the life-span and reduce the maintenance costs. 

In this project, I will be looking specifically at a phenomenon known as a "short-period oscillation". This is one of the two longitudinal dynamic modes that occur on every aircraft, the other being a "phugoid oscillation". Short-period oscillations are characterised by the rapid decay and high frequency. A typical SPO can settle within a second and can occurs at 1-2Hz. A pilot can induce a SPO by a sharp pitch input. They can also naturally occur due to a gust (impulsive increase/decrease in airspeed and/or angle of attack). 

A UAV control system that does not have a SPO model is vulnerable to growing instability. It can act to reinforce the oscillation instead of canceling them. Thus it is important to systematically analyse and design a model that can handle a SPO.  
\newpage
%----------------------------------------------------------------------------------------
%	ACKNOWLEDGEMENTS
%----------------------------------------------------------------------------------------
\begingroup
\let\clearpage\relax
\let\cleardoublepage\relax
\begin{acknowledgements}
\addchaptertocentry{\acknowledgementname} % Add the acknowledgements to the table of contents
I would like to acknowledge my supervisor, Dr T. Glyn Thomas who provided me with guidance every step of the way and supported my ideas that enabled me to complete this project.\ldots

\end{acknowledgements}
\endgroup
\newpage
%----------------------------------------------------------------------------------------
%	ABBREVIATIONS
%----------------------------------------------------------------------------------------
\begingroup
\let\clearpage\relax
\let\cleardoublepage\relax
\begin{abbreviations}{ll} % Include a list of abbreviations (a table of two columns)

\textbf{LAH} & \textbf{L}ist \textbf{A}bbreviations \textbf{H}ere\\
\textbf{WSF} & \textbf{W}hat (it) \textbf{S}tands \textbf{F}or\\

\end{abbreviations}
\endgroup
\newpage
%----------------------------------------------------------------------------------------
%	PHYSICAL CONSTANTS/OTHER DEFINITIONS
%----------------------------------------------------------------------------------------
\begingroup
\let\clearpage\relax
\let\cleardoublepage\relax
\begin{constants}{lr@{${}={}$}l} % The list of physical constants is a three column table

% The \SI{}{} command is provided by the siunitx package, see its documentation for instructions on how to use it

Speed of Light & $c_{0}$ & \SI{2.99792458e8}{\meter\per\second} (exact)\\
%Constant Name & $Symbol$ & $Constant Value$ with units\\

\end{constants}
\endgroup
\newpage
%----------------------------------------------------------------------------------------
%	SYMBOLS
%----------------------------------------------------------------------------------------
\begingroup
\let\clearpage\relax
\let\cleardoublepage\relax
\begin{symbols}{lll} % Include a list of Symbols (a three column table)

$a$ & distance & \si{\meter} \\
$P$ & power & \si{\watt} (\si{\joule\per\second}) \\
%Symbol & Name & Unit \\

\addlinespace % Gap to separate the Roman symbols from the Greek

$\omega$ & angular frequency & \si{\radian} \\

\end{symbols}
\endgroup
\newpage
\tableofcontents

%----------------------------------------------------------------------------------------
%	THESIS CONTENT - CHAPTERS
%----------------------------------------------------------------------------------------

\mainmatter % Begin numeric (1,2,3...) page numbering

\pagestyle{thesis} % Return the page headers back to the "thesis" style

% Include the chapters of the thesis as separate files from the Chapters folder
% Uncomment the lines as you write the chapters

\chapter{Background}
\chapter{Setting up}
\newpage
%----------------------------------------------------------------------------------------
%	THESIS CONTENT - APPENDICES
%----------------------------------------------------------------------------------------

\appendix % Cue to tell LaTeX that the following "chapters" are Appendices

% Include the appendices of the thesis as separate files from the Appendices folder
% Uncomment the lines as you write the Appendices

\include{Appendices/AppendixA}
%\include{Appendices/AppendixB}
%\include{Appendices/AppendixC}

%----------------------------------------------------------------------------------------
%	BIBLIOGRAPHY
%----------------------------------------------------------------------------------------
\printbibliography[heading=bibintoc]
%----------------------------------------------------------------------------------------

\end{document}  
